\begin{abstract}

本文通过几何分析、数理统计、信号处理、计算机辅助处理数据等方法,建立了存在测量误差情况下二维CT扫描参数标定的模型,通过模板标定了系统的扫描线间隔、旋转角度、旋转中心等参量。借助这些标定的参量, 我们完成了从扫描数据中重建物体几何形状、位置、吸收率等信息的任务。最后,我们定量分析了原有模型的缺陷,设计了新的标定模型以提高其稳定性与精确度,并对我们的工作进行了整体评价。

对于扫描线间隔的计算,利用数据中无重合的圆条带部分,可求扫描线被圆所截得的弦长。由于相邻扫描线之间间距相同,确定一条扫描线后可利用该间距得到其余扫描线的方程。后利用弦长及垂径定理可得到一个二元线性方程,其中一个变元为待求扫描线间距的平方。由于一个角度下有多条扫描线与圆相交,可以列出一组超定的方程组,后利用二元线性回归即可求得扫描线间距。由于有多个旋转角度,我们对于每个旋转角度都计算一次回归,最后对得到的数据求平均值,即得扫描线间距为$0.2766\,\si{mm}$。

旋转角度的计算与上述过程类似,利用扫描线被椭圆截得的弦长,可列出一组含其中一条截距以及斜率的非线性方程,对相邻方程进行求解可得到一组斜率值,均值作为当前角度的正切值。在旋转角接近$90^\circ$时会出现数值上的问题,此时选用斜率倒数以及$x$轴截距作为变元即可。求得系统与托盘的初始夹角为$\theta_0=29.6422^\circ$。具体的180个旋转角可见附录中表\ref{table:roration_degrees}。

在特定角度下,由线间隔可以精确地确定圆心投影位置,再加上系统旋转后圆心在旋转坐标系中坐标的变化等约束,可构造多个方程组求解旋转坐标系的原点坐标。对所有值取平均即得到旋转中心$(-9.2383, 6.2663)\si{mm}$,位于托盘中心的略左上方处。

图像的重建问题等价于逆 Radon 变换。一种算法为像素驱动的直接反投影,即计算待求像素在各个旋转角上的投影位置。其最终吸收率为其各个旋转角上投影位置上的吸收值之和。此外,可利用线性插值来将吸收值连续化。由于直接反投影带来图像模糊等问题,我们利用Fourier中心切片定理推导出了一种能较精确地还原原始信息的滤波反投影算法,即将各角度下的投影值与R-L滤波函数卷积后的值进行直接反投影。实验证明此方法是优秀的 ,恢复的图像可见附录中图\ref{fig:4_filter_copy},题中要求的10点吸收率可见附录中表\ref{table:absort_rate},详细各点吸收率可见附件。

此外,在测量值有误差的假设下,我们证明了计算出的扫描线间隔的精确度和圆模型的半径成正相关;以及旋转角度的求解结果在扫描线与椭圆的轴接近平行的临界情况下变得不稳定。为此,可利用多模板多次扫描的方法将圆与椭圆的信息解耦合,并且增大圆的半径来提高计算 扫描线间隔的精确度。对于旋转角度的不稳定性,可用椭圆在坐标系中不同的放置方式获取多组数据,最后取最小方差的角度值作为真实值。

最后,我们回顾了整个模型,分析其优点和不足之处,并对其将来的应用提出展望。

\keywords{医学影像\quad 图像重建\quad 参数标定\quad Radon变换\quad 信号处理}
\end{abstract}
