\section{模型评价与展望}
在这次建模过程中,我们成功完成了题目所要求的标定系统参数与重建扫描图像的任务,并定量分析原有模型的不足,提出了在理论上更优的标定模型。

我们的整体工作突出之处有:
\begin{itemize}
  \item 在标定参数的过程中充分利用题目给出的数据与几何关系构成方程,求解得到的参数值较精确,统计意义上较为优秀
  \item 在重建图像的过程中选择了恰当的投影方法与滤波函数,恢复出的图像轮廓清晰,在样本上的恢复与原数据吻合程度较高
  \item 在模型分析过程中充分考虑可能的测量误差,定量地建立了样本数量、系统状态与误差间的数学关系
  \item 提出的新模型在标定过程中使用的算法较简单,且得到的数据精度更高
\end{itemize}

不足之处有:
\begin{itemize}
  \item 求解参数过程使用的关系式较为复杂,且数据量大,导致求解过程略显缓慢
  \item 使用的R-L滤波函数会在空域中造成震荡响应,降低了吸收率较低的介质区域的信噪比
  \item 提出的新模型需要较多次扫描得到的数据,操作较为繁琐,且对模板放置位置精度要求较高
\end{itemize}

此模型未来可能有的进一步发展包括且不限于:
\begin{itemize}
  \item 调整算法以适应接收器间距不再严格相等的情况,消除系统带来的误差
  \item 考虑射线在空气等非被测介质中的损耗及可能的散射、反射等现象带来的干扰
  \item 将标定与重建算法从二维空间推广到三维甚至更高维度(含时)的空间
  \item 优化计算复杂度,减少待解方程数量,将数值运算问题转化为较易解决的统计学问题
\end{itemize}