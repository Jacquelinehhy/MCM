\section{问题重述}
CT是医学、生物领域重要的成像手段,可以有效获取样品的 内部结构信息。题目给出了一种典型的二维CT系统,平行入射的X射线垂直于探测器平面,经过样品吸收后被等距排列的512个探测器单元接收。每次扫描过程中,整个发射-接受系统绕着旋转中心逆时针旋转180次,得到180组接收信息。

由于现实中CT系统存在误差,故需要借助已知结构的样品(模板)对系统的参数进行标定,并据此对未知结构的样品进行成像。

本题目要求建立数学模型,解决以下问题:

\begin{enumerate}
  \item 根据已知的模板几何信息和扫描结果,确定CT系统的旋转中心、探测器单元距离与180个旋转方向
  \item 根据上述标定的信息,从扫描数据中恢复出未知介质的位置、形状、吸收率
  \item 分析上述标定的精度和稳定性,设计新模板、建立标定模型以改进这些指标
\end{enumerate}

\section{问题分析}

问题1要求从已知的几何信息和扫描结果中恢复出系统的一系列参数。根据比尔-朗伯定律可知,在吸收率恒定的情况下,X射线的吸收值(即给出的测量值)与穿过物体的厚度成正比,故本小题中吸收值也可以作为物体厚度的一种度量尺寸;又由于两条X射线之间的间距恒定,故其亦可作为度量单位。再加上题目本身给出的SI制与$256\times256$两种度量,共有四种尺度。因而解决本题时尤其要注意各个尺度之间的换算和处理。我们使用每个圆中不同扫描线的长度处理得到扫描线宽,最后所有值平均得到真实线宽(即接收器间距)。由于圆直径最长,而数据中属于圆的吸收值部分出现多次明显的最大值,据此可得到圆直径在吸收值尺度下的值,也就得到了吸收值尺度与真实尺度的换算关系。对于不重叠部分,通过扫描线长处理可得到实际圆心的投影坐标。而后,在旋转中心建立另一坐标系,通过椭圆弦长与坐标系旋转的位置关系等几何约束列出方程组,可确定每一组数据对应的旋转角度(即180个方向)。再通过坐标系旋转后圆心坐标的变化等约束,可求解圆心在旋转坐标系中的坐标,也就确定了旋转中心相对托盘的位置。

问题2、3要求根据问题1中得到的参数,从扫描数据中重建介质的二维图像,即重建各点的吸收率信息,这显然是对第一题获得信息的一种检验。在求得旋转中心位置后,很容易在旋转坐标系和静止坐标系间进行变换。最简单的思路是对于每个旋转角度,将投影信息反向投影到原坐标系中;最后所有角度的信息叠加,就能得到大致的原图像。但显然这种方法的的准确度不够高,会造成各点处强度值的平均化,因此需要通过更精细的处理方法获得更接近的原始图像。

问题4要求分析问题1中参数标定的精度与稳定性,以及设计新模板、建立新模型,题目要求比较开放。这要求我们在完成问题1后,对数据的处理过程进行定量的分析。除了题中要求的接收器间距、转动角度、转动中心位置等重要物理量外,还应该对求解这些量的中间过程和理论推导进行研究。对同一个量多次估计的方差、关于同一量的超定方程组解的情况等都可以成为重要的量化指标。同样地,在设计新模板和新的标定模型时,也应当着重关注这些指标,以尽可能低的成本、尽可能简单的方式,做到精确而稳定地标定系统的各项参数。

\section{模型假设与符号约定}

在本题的求解过程中,我们做出以下基本假设:

\begin{assumption}
探测平台的180个旋转方向在每次扫描时是不变的。
\end{assumption}

\begin{assumption}
X射线在非被测介质中无损耗传播,不发生衰减、吸收。
\end{assumption}

\begin{assumption}
X射线在任何情况下中不发生散射、反射。
\end{assumption}

\begin{assumption}
所有的X射线发射器与接收器均没有损坏。
\end{assumption}

在建模与求解过程的中,我们使用了表\ref{table:symbol}中定义的一些记号。

\begin{table}[htbp]
\centering
\caption{本文中使用的记号}
\label{table:symbol}
\begin{tabular}{@{}ll@{}}
\toprule
符号          & 说明                           \\ \midrule
$s$         & 下标后缀:以接收器间隔为尺度的长度量           \\
$i$         & 下标后缀:以吸收量为尺度的长度量             \\
$\mu(x,y)$  & 介质在$(x,y)$点的衰减系数分布函数          \\
$F(u,v), F(\rho,\theta)$  & 衰减系数分布进行二维Fourier变换的结果的直角坐标与极坐标形式 \\
$g_\theta(R), \hat{g}_\theta(R)$ & 探测系统转动角度为$\theta$时位于坐标$R$处的投影的真实值与滤波值 \\
$\mathcal F_1, \mathcal F_1^{-1}$ & 一维空间中的Fourier变换与逆变换算子\\
$xoy$       & 以托盘中心为原点建立的直角坐标系             \\
$XOY$       & 以探测系统旋转中心为原点建立的坐标系           \\
$l$         & 两个相邻接收器间的距离                  \\
$R$         & 模型中小圆的半径                      \\
$a, b$      & 模型中椭圆$ox$与$oy$方向的半轴长         \\
$L_m$       & 线长:通过介质的第$m$条射线处的介质厚度        \\
$D_m$       & 线距:通过介质的第$m$条射线与最靠近第$0$条线的圆切线的距离    \\
$\theta$    & $oxy$坐标系相对于其原始状态逆时针旋转的角度     \\
$\omega, k$ & 扫描线与$oxy$坐标系$ox$正方向的夹角及其对应斜率 \\
$C(X_0,Y_0)$   & $XOY$坐标系初始状态下小圆圆心的坐标 \\ \bottomrule
\end{tabular}
\end{table} 